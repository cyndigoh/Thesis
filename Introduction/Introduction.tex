\chapter{Introduction}
\label{ch:Introduction}
\textit{This chapter presents the aims of this thesis, and provides an overview of pre-existing knowledge relevant to these goals}

% 8595 text, 150 headers

\startcontents[chapters]{\vspace{-1.4cm}}
\singlespacing
\printcontents[chapters]{}{1}{\section*{ }\setcounter{tocdepth}{1}}
\doublespacing
\vspace{0.5cm}

The overall objective of this thesis is to 

\section{Genetics and gene expression}

The human genome was first sequenced completely by the Human Genome Project Consortium in 2003 \parencite{Collins2003}. 

\subsubsection{Types of genetic variation}

A polymorphism is a genetic locus that exists in multiple forms, or alleles, at appreciable frequencies in a population (e.g. minor allele $\geq1\%$). 

\section{Specific aims and objectives}

The aims of this thesis are to:

\begin{itemize}[leftmargin=*]

\item	\textbf{Investigate the host transcriptome in sepsis of different sources 
(\hyperref[ch:Results1]{Ch. 3})} \\
	The two most common causes of sepsis in the UK are community acquired pneumonia (CAP) and faecal peritonitis (FP). The host transcriptome in sepsis due to FP has not yet been described, and the clinical differences observed between CAP and FP suggest that variation may also be observed in the molecular response to disease. I aim to:

		\begin{enumerate}
			\item describe the host transcriptome in sepsis due to FP and CAP for peripheral blood leukocytes
			\item directly compare gene expression between CAP and FP patients
			\item identify temporal changes in gene expression in CAP and FP
		\end{enumerate}

	\item  \textbf{Compare the sepsis transcriptomic response to related conditions 
	(\hyperref[ch:Results1]{Ch. 3})} \\
The sepsis response is related to the systemic inflammatory response to insults such as surgery and trauma. Comparison of the infectious and sterile responses might be aid interpretation of the biological processes involved. I will therefore:
		\begin{enumerate}
				\item define the transcriptomic response in:
					\begin{enumerate}
							\item sepsis due to CAP and FP, in contrast to non-septic samples
							\item traumatic injury, in contrast to healthy volunteers
							\item cardiac surgery, using pre- and post-operative samples
						\end{enumerate}
					\item compare these responses across conditions
				\end{enumerate}
		
\end{itemize} 
